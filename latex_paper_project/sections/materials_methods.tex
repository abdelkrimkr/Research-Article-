\section{Matériels et Méthodes}
\subsection{Animaux d'expérience}
Des souris mâles adultes (par exemple, de souche C57BL/6), âgées de 8 semaines et pesant environ 20-25g, ont été utilisées pour cette étude. Elles ont été acclimatées pendant une semaine dans des conditions standard de laboratoire (température 22 ± 2°C, cycle lumière/obscurité de 12h/12h) avec accès libre à l'eau et à une nourriture standard.

\subsection{Induction de l'hyperhomocystéinémie}
L'hyperhomocystéinémie a été induite par un régime alimentaire enrichi en méthionine pendant une période de 4 semaines, suivant un protocole adapté de la littérature. Un groupe témoin a reçu un régime standard.

\subsection{Préparation de l'extrait de Stachys mialhesi}
Les parties aériennes de Stachys mialhesi ont été récoltées, séchées à l'ombre, puis réduites en poudre. L'extrait butanolique a été préparé par macération de la poudre dans du butanol, suivi d'une filtration et d'une évaporation du solvant sous pression réduite. L'extrait sec a été conservé à 4°C jusqu'à son utilisation.

\subsection{Traitements}
Les souris hyperhomocystéinémiques ont été divisées en plusieurs groupes : un groupe HHcy non traité, un groupe traité avec l'extrait butanolique de S. mialhesi (dose à spécifier), un groupe traité avec la vitamine B9 (dose à spécifier), et un groupe traité avec la vitamine B12 (dose à spécifier). Les traitements ont été administrés quotidiennement par gavage oral pendant X semaines.

\subsection{Prélèvements et analyses biochimiques}
À la fin de la période de traitement, les souris ont été anesthésiées et le sang a été collecté par ponction cardiaque. Le sérum a été séparé pour le dosage de l'homocystéine, du cholestérol total, des triglycérides, et d'autres marqueurs pertinents (à spécifier) en utilisant des kits commerciaux et un analyseur biochimique.

\subsection{Analyse histologique de l'aorte}
Après le prélèvement sanguin, les aortes thoraciques ont été prélevées, fixées dans du formol tamponné à 10%, puis incluses en paraffine. Des coupes de 5 µm d'épaisseur ont été réalisées et colorées à l'Hématoxyline-Éosine (H&E) et par d'autres colorations spécifiques (ex: coloration de Verhoeff pour les fibres élastiques) pour examiner la morphologie générale et l'intégrité structurelle.
