\section{Introduction}
L'hyperhomocystéinémie (HHcy) est un facteur de risque reconnu pour les maladies cardiovasculaires \cite{placeholder_hcy}. Elle est caractérisée par une élévation anormale du taux d'homocystéine dans le sang. Plusieurs études ont exploré des moyens de la prévenir ou de la traiter, notamment par des approches nutritionnelles.
La plante Stachys mialhesi, endémique de certaines régions, a été traditionnellement utilisée pour diverses affections, mais ses effets spécifiques sur l'HHcy et la structure vasculaire sont peu documentés.
Les vitamines B9 (acide folique) et B12 (cobalamine) jouent un rôle crucial dans le métabolisme de l'homocystéine et sont souvent utilisées pour réduire ses niveaux.
Cette étude vise à comparer l'effet d'un extrait butanolique de Stachys mialhesi avec celui des vitamines B9 et B12 sur certains paramètres biochimiques sériques et sur la structure histologique de l'aorte chez des souris rendues hyperhomocystéinémiques.
